%%%%%%%%%%%%%%%%%%%%%%%%%%%%%%%%%%%%%%%%%%%%%%%%%%%%%%%%%%%%%%%%%%%%%
%   GUP ver. 1.0      Sodankyla  8. June 1990                       %
%   copyright Markku Lehtinen                                       %
%%%%%%%%%%%%%%%%%%%%%%%%%%%%%%%%%%%%%%%%%%%%%%%%%%%%%%%%%%%%%%%%%%%%%
% Macros to calculate units in the case where they are to be determined
% from axis dimensions (in mm) and limits (in user coordinates).
% If \xwidth or \ywidth is zero, no units are calculated and the user
% must himself specify the \xunit or \yunit parameters
\newdimen\dima\newdimen\dimb\newdimen\dimc\newdimen\helpdim
\def\s{ }

% Finds labellable tick values:
\def\calculateaxislimits{
\def\xlastlabel{}\def\xunits{}\def\yunits{}
% xamax, xlastlabel
\dima=\xmax pt\advance\dima by 0.00001 pt
\dimb=-\xnumberinterval pt
\loop\ifdim\dima<0pt
  \advance\dima by \xnumberinterval pt \advance\dimb by -\xnumberinterval pt
\repeat
\loop\ifdim\dima>0pt
  \advance\dima by -\xnumberinterval pt \advance\dimb by \xnumberinterval pt
\repeat
\placevalueinpts of <\dimb> in {\xamax}
\advance \dimb by -\xnumberinterval pt
\placevalueinpts of <\dimb> in {\xlastlabel} 
% xamin:
\dima=\xmin pt\advance\dima by -0.00001pt\dimb=0pt
\loop\ifdim\dima<0pt
  \advance\dima by \xnumberinterval pt \advance\dimb by -\xnumberinterval pt
\repeat
\loop\ifdim\dima>0pt
  \advance\dima by -\xnumberinterval pt \advance\dimb by \xnumberinterval pt
\repeat
\placevalueinpts of <\dimb> in {\xamin}
% yamax:
\def\ynumberinterval{1 }
\dima=\ymax pt\advance\dima by 0.00001pt
\dimb=-\ynumberinterval pt
\loop\ifdim\dima<0pt
  \advance\dima by \ynumberinterval pt \advance\dimb by -\ynumberinterval pt
\repeat
\loop\ifdim\dima>0pt
  \advance\dima by -\ynumberinterval pt \advance\dimb by \ynumberinterval pt
\repeat
\placevalueinpts of <\dimb> in {\yamax}
% yamin:
\dima=\ymin pt\advance\dima by -0.00001pt \dimb=0pt
\loop\ifdim\dima<0pt
  \advance\dima by \ynumberinterval pt \advance\dimb by -\ynumberinterval pt
\repeat
\loop\ifdim\dima>0pt
  \advance\dima by -\ynumberinterval pt \advance\dimb by \ynumberinterval pt
\repeat
\placevalueinpts of <\dimb> in {\yamin}
\ifdim \dimb < 1pt \def\yamin{1 } \fi % special handling for y=0
} % end of calculateaxislimits

% Calculates scales of the picture.
% If \xwidth is defined, resulting picture size is specified by it
% else picture size is calculated from \xunit and (\xmax-\xmin)
\def\calcunits{
% xunit:
\ifnum\xwidth =0\else
\dima=\xmin pt\dimb=\xmax pt\advance\dimb by -\dima
\dima=\xwidth pt\Divide <\dima> by <\dimb> forming  <\dimc> 
\placevalueinpts of <\dimc> in {\xunit} \fi %\show\xunits
\ifnum \ywidth =0\else
\dima=\ymin pt\dimb=\ymax pt\advance\dimb by -\dima
\dima=\ywidth pt\Divide <\dima> by <\dimb> forming  <\dimc> 
\placevalueinpts of <\dimc> in {\yunit} \fi
}

% Macro for giving values in a compact way:
\def\xticks #1 #2 #3 #4 #5 #6 #7 {%
\def\xmin{#1}\def\xamin{#2}%
\def\xnumberinterval{#3}\def\xtickinterval{#4}%
\def\xlastlabel{#5}\def\xamax{#6}\def\xmax{#7}}

% Macro that transfers relevant TeX parameters to PS interpreter so
% that necessary scaling calculations can be made in PS.
\def\parsfromTeXtoPS{
/xunit{\xunit\s 1000 div}def/yunit{\yunit\s }def
/xref{\xmin\s 1000 mul}def/yref{\ymin\s }def
/xmin{\xmin\s 1000 mul}def/xmax{\xmax\s 1000 mul}def
/ymin{\ymin\s }def/ymax{\ymax\s }def
}

% Macro to set up PiCTeX world in terms of the picture scale variables used
\def\setdefaultsystem{% defines standard PiCTeX world in variables defined
\setcoordinatesystem units <\xunit mm,\yunit mm> 
     point at {\xmin} {\ymin} %point inside page
\setplotarea x from {\xmin} to {\xmax}, y from {\ymin} to {\ymax}
}

% Macro that can be redefined by the user to add some additional elements
% to the picture (e.g. axis ticks, titles etc.)
\def\addtopict{}
\def\addtoPS{}

% Macro that draws a combined PiCTeX / PostScript picture 
% so that PiCTeX coordinates can be used in Postscript code
\def\GUPPiC{
%\vskip 3mm plus 3mm minus 2mm
\calculateaxislimits
\calcunits
\beginpicture
\setdefaultsystem
\GUPPiCspecials
\grid 1 1
\addtopict
\endpicture
%\vskip 3mm plus 3mm minus 2mm
}
\def\addtopict{}
