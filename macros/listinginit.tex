%%%%%%%%%%%%%%%%%%%%%%%%%%%%%%%%%%%%%%%%%%%%%%%%%%%%%%%%%%%%%%%%%%%%%
%   GUP ver. 1.0      Sodankyla  8. June 1990                       %
%   copyright Markku Lehtinen                                       %
%%%%%%%%%%%%%%%%%%%%%%%%%%%%%%%%%%%%%%%%%%%%%%%%%%%%%%%%%%%%%%%%%%%%%
\newcount\hour \newcount\hours  \newcount\minute
\def\clock#1{\hour=#1\hours=#1\minute=#1\divide\hours by 60 \hour=\hours 
\multiply\hours by 60 \advance\minute by -\hours  
\ifnum\minute<10 \number\hour.0\number\minute\else					        
\number\hour.\number\minute\fi}

\def\today{\ifcase\month\or January \or February\or March\or April\or May
\or June\or July\or August\or September\or October\or November\or December\fi
\space\number\day, \number\year}

\headline={{\sevensy $\clubsuit$\sevenrm~  GUISDAP documentation: {\jobname.tex}
\hskip 0.5truecm \GUPlocation \hskip 0.5truecm  \today\ at
\clock\time\hfill} }

% fonts for different headings                                     		
\font\chfont=cmbx12 scaled \magstep1 % main chapters font
\font\sfont=cmr10 scaled\magstep1    % subchapters font
\font\sbsfont=cmr10                  % subsubn font
		
% rows for the table of contents are made here
\def\lead{\leaders\hbox to 12pt{\hfill.\hfill}\hfill}
\def\entry:#1\page#2\\{\hbox to\hsize{{\hskip30pt\sfont#1}\hfill#2}} 

\def\listting#1{%
%\write2{#1 \number\pageno}
{\baselineskip=8pt\parskip=0pt\parindent=1cm\line{}%
  {#1\par}
  \par\begingroup\setupverbatim\input#1 \endgroup\par\medskip}
		\catcode`\_=8\endgroup
		
% save name to toc file
\edef\save{\string\entry:#1%
  \string\page\noexpand\folio\string\\}%
\write\toc\expandafter{\save}

}
\def\listing{\begingroup\par\catcode`\_=\other\catcode`\@=\other\tt\listting}

\vsize=9.5in\parindent=0pc\parskip=6pt

% macros for table of contents

\def\tochead{\centerline{\chfont Table of contents}}
\def\ch#1:#2\page#3\\{\vskip5pt minus3pt\leftline{\chfont#1.#2}}
\def\se#1:#2\page#3\\{\hbox to\hsize{{\sfont\quad
#1.#2}\lead#3}}
 \def\sbs#1:#2\page#3\\{\hbox to\hsize{{\sbsfont\qquad
#1.#2}\lead#3}}
%chapter without number 
\def\oth:#1\page#2\\{\hbox to\hsize{{\chfont#1}\lead#2}} 

\def\str:#1\\{\leftline{\chfont#1}}
 
\newcount\chnum  % chapter number
\chnum=0
\newcount\snum   % subnumber
\newcount\sbsnum % subsubnumber!

% stuff for toc without a page number
\def\strange#1{
\edef\save{\string\str:#1%
  \string\\}%
\write\toc\expandafter{\save}}


\def\other#1{
\leftline{\chfont  #1}
\edef\save{\string\oth:#1%
  \string\page\noexpand\folio \string\\}%
\write\toc\expandafter{\save}}

\def\chapter#1{\global\advance\chnum by1
\global\snum=0 \sbsnum=0
\leftline{\chfont \the\chnum. #1}
\edef\save{\string\ch\the\chnum:#1%
  \string\page\noexpand\folio\string\\}%
\write\toc\expandafter{\save}}

\def\section#1{%
\global\advance\snum by1 \sbsnum=0
{\leftline{\sfont \the\chnum.\the\snum. #1}}
\edef\save{%
  \string\se\the\chnum.\the\snum:% 
  #1\string\page\noexpand\folio\string\\}%
\write\toc\expandafter{\save}}

\def\sbsection#1{%
\global\advance\sbsnum by1
{\leftline{\sbsfont \the\chnum.\the\snum.\the\sbsnum. #1 }}
\edef\save{%
  \string\sbs
    \the\chnum.\the\snum.\the\sbsnum:% 
 #1\string\page\noexpand\folio
  \string\\}%
\write\toc\expandafter{\save}}

\def\readtoc{
\immediate\closeout\toc
\immediate\openin\toc=\jobname.toc
\ifeof\toc
   \message{! No file \jobname.toc;}
\else
   \tochead 
   \input\jobname.toc \vfill\eject
\fi
\immediate\closein\toc}
% start to write the toc file
\newwrite\toc
\immediate\openout\toc=\jobname.toc


% ----------------------------------------------------------------
% Making INDEX file
\font\indfont=cmr8 scaled \magstep0
\def\ind:#1\page#2\\{\hbox to\hsize{{\indfont#1}\hfill#2}} 

\def\index#1{#1
\edef\save{\string\ind:#1%
  \string\page\noexpand\folio \string\\}%
\write\indfile\expandafter{\save}}
\def\indhead{\centerline{\indfont Index}}


\def\readindfile{
\immediate\closeout\indfile
\immediate\openin\indfile=\jobname.ind
\ifeof\indfile
   \message{! No file \jobname.ind;}
\else
   \indhead 
   \input\jobname.ind
			\vfill\eject
\fi
\immediate\closein\ind}

% open index file:
\newwrite\indfile
\immediate\openout\indfile=\jobname.ind

% -----------------------------------------------------------------


%Fljande macron anvnds d man vill f ut \TeX-kommandon
%i klartext.  @text@ ger text i fonten \tt med alla kontroll-
%tecken och dylika bevarade (\TeX\ frstr sig endast p kommandot
%@ i denna mode). Man kan ocks utnyttja \begintt ... \endtt fr
%att beskriva kommandon, d r ocks \obeylines och \obeyspaces
%aktiva. Om man vill anvnda vissa \TeX-kommandon men vill
%ocks displaya ngot i en-kolumn-tabell-form kan man anvnda
%\begindisplay ...\cr ...\cr \enddisplay
%           Om du vill anvnda \begintt...\endtt fr text som inte
%       ryms p en sida, kan du ocks anvnda \beginlines...\endlines.
%       Varje rad mste d brjas och avslutas med kommandot @.
\newcount\linenumber
\def\clearlinenumber{\linenumber=2}
\def\lineno{\advance\linenumber by 1$_{\number\linenumber}$}
\clearlinenumber
%
\newskip\verbatimindent \verbatimindent=1.5cm
\newskip\ttglue{\tt \global\ttglue=.5em plus.25em minus.15em}
\def\@{\char'100 }
%
\outer\def\begindisplay{\obeylines\startdisplay}
{\obeylines\gdef\startdisplay#1
    {\catcode`\^^M=5$$#1\halign\bgroup&\hskip\verbatimindent##\hfil\cr}}
\outer\def\enddisplay{\crcr\egroup$$}
%
\chardef\other=12
\def\ttverbatim{\begingroup 
  \ifmmode%
 		  \catcode`\_=\other\mathcode`\_="0274\tt 
%    \catcode`\_=\active
		\else
				\catcode`\\=\other \catcode`\{=\other
    \catcode`\}=\other \catcode`\$=\other \catcode`\&=\other
    \catcode`\#=\other \catcode`\%=\other \catcode`\~=\other
    \catcode`\_=\other \catcode`\^=\other\catcode`\*=\other\catcode`\==\other
    \obeyspaces \obeylines \tt
		\fi}
{\obeyspaces\gdef {\ }}
%
\outer\def\begintt{$$\let\par=\endgraf \ttverbatim \parskip=0pt
    \catcode`\@=0 \parindent=\verbatimindent \rightskip=-5pc \ttfinish}
{\catcode`\@=0 @catcode`@\=\other % @ is temporary escape character
    @obeylines % end of line is active
    @gdef@ttfinish#1^^M#2\endtt{#1@vbox{#2}@endgroup$$}}
%
\def\aaa{{\tt\char66 }}
%\mathcode`\@=\active
\catcode`\@=\active
{\obeylines\gdef@{\ttverbatim\spaceskip=\ttglue\let^^M=\ \let@=\endgroup}}
%
\def\beginlines{\par\begingroup\nobreak\medskip\parindent=0pt
   \kern1pt\nobreak \obeylines \everypar{\strut}}
\def\endlines{\kern1pt\endgroup\medbreak\noindent}
%
\def\verbstyle{\parindent=3pc\parskip=0pt}

\font\ttsmall=cmtt8
%\font\numbering=Times at 5pt
\font\numbering=cmr5
\def\uncatcodesspecial{\def\do##1{\catcode`##1=12 }\dospecials}
\newcount\lineno
\def\setupverbatim{\ttsmall\lineno=0
  \def\par{\leavevmode\endgraf} \catcode`\`=\active
  \obeylines \uncatcodesspecial \obeyspaces
  \everypar{\advance \lineno by 1\hbox to1cm{\numbering\the\lineno\hfill}}}
{\obeyspaces\global\let =\ } 
  {\catcode`\`=\active \gdef`{\relax\lq}}
		
